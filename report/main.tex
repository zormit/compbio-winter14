\documentclass[a4paper, 11pt, parskip=half]{scrartcl}

\usepackage[english]{babel}
\usepackage[utf8]{inputenc}
\usepackage[T1]{fontenc}

\newcommand{\titleString}{Use of noisy constraints in Protein Structure
Prediction}
\newcommand{\groupName}{Methionine}

\usepackage[hidelinks,
            pdfauthor={\groupName},
            pdftitle={\titleString},
            pdfsubject={Computational Biology}]{hyperref}

\usepackage[utf8]{inputenc}
\usepackage[pdftex]{graphicx}
\usepackage{float}

\usepackage{authblk}

\title{\titleString}

\author{David Lassner}
\author{Moritz Neeb}
\affil{\groupName}

\graphicspath{{./plots/}}

\begin{document}

\maketitle

% According to https://www.isis.tu-berlin.de/2.0/mod/assign/view.php?id=84719:
% try to focus on the analysis of your results and the insights and future work you got from the projects.

%\section*{Abstract}
%TODO should we have an abstract? i dont think so. this is just a reminder to discuss it.

\section{Introduction}

\subsection{Background}

Ab initio protein structure prediction is hard.
%TODO reformulate previous sentence more scientific
To make it easier, ab initio protein structure prediction
can be enhanced by \emph{constraints}.
A constraints is an information about the proximity of residues in the native protein.
This information can be used to guide the prediction and possibly improve it.
Typically, a set of constraints is used together to guide the search.
These constraints can for example be a result of protein contact prediction,
which has to be run as a preliminary step.
%TODO can you say "has to be run"?
%TODO reference to Schneider/Brock 2014

\subsubsection*{Definitions}
\begin{description}
	\item[fulfilled constraints] are constraints that are within the range of $0$ and $8$ angstrom in a protein (decoy or native).
	\item[native constraints] are the fulfilled constraints in the native protein,
	i.e. those that are expected to guide the prediction towards the native structure.
\end{description}

\subsection{Goal and Approach of this Project}
%TODO split goal <-> approach??

In this project, noisy constraints were used.
That means, most of the constraints are misleading (i.e. not native),
as the contact prediction step is inaccurate.
%TODO reference to amount, e.g. 63 natives of 225 for 2h3jA
The challenge is to make use of the constraint set despite their noisiness.
The approach in this project basically contained three steps:
\begin{enumerate}
	%TODO conc. wording: group or subset? In the code, we said group,
	% but I think subset is better here.
	\item Constraint subsets
	were generated through a heuristic from which it was expected to get
	a high percentage of native constraints.

	\item The protein structure was predicted with these subsets of constraints as input.

	\item The information available (i.e. decoy scores and fulfilled constraints)
	was exploited to predict whether a subset
	contains native constraints or not.
\end{enumerate}

\subsubsection*{Subset Generation}
As it is computationally not feasible to run the structure prediction
for every possible subset of the noisy constraints,
a heuristic has to be used, to get subsets that are promising.

In this project, subsets were generated based on the information of secondary structure:
Every constraint contains two residues that it constrains.
%TODO wortspielkasse!? or:
% How to explain the concept of contraint->two residues -> two secondary structures -> one subset
Each residue position in the sequence belongs to a specified secondary structure.
A subset consists of all constraints that link one secondary structure to another.
%TODO example and figure

%TODO explain reasons, why we did this: cf. milestone 1:
% compatibility, parallel/crossing constraints etc.

%\subsection{Penalty of Constraints}
%TODO Describe scoring and rescoring?

\subsection{Technology Used}

The project's software consists of python and shell scripts that are used to
handle the constraint analysis and subset generation,
calls to Rosetta, overall analysis and plotting.

\section{Analysis and Results}
The following assumptions were made:
1) The native constraints produce better predictions.
%TODO this can be seen in the results, so this is not an assumption.
2) The chance of a constraint to be native is higher if it is compatible
with other constraints.

\section{Insights}

\section{Future Work}

\subsection{Iterative Algorithm}

\subsection{Predict non Native Constraint Subsets}

\subsection{Adapting Weights}

\subsection{Other Heuristics}
\end{document}
